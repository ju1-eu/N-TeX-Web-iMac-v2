%% Cornell Methode
%% Duplex Druck
\documentclass[a4paper,11pt]{scrartcl}
\usepackage[utf8]{inputenc} 
\usepackage[T1]{fontenc}
\usepackage[ngerman]{babel}
\usepackage{calc} 
\pagestyle{empty}
\usepackage{geometry}
\geometry{noheadfoot,nomarginpar,top=0.5cm,bottom=0.3cm,left=1.6cm,right=1.6cm}
\usepackage{tikz} 

\newlength{\wholeboxwd}
\setlength{\wholeboxwd}{0.98\textwidth}
\newlength{\wholeboxht}
\setlength{\wholeboxht}{0.95\textheight}
%
\newlength{\topwd}
\setlength{\topwd}{0.9\wholeboxwd}%Vorlesung/Thema/Datum
\newlength{\cuewd}
\setlength{\cuewd}{0.32\wholeboxwd}%6cm Stichworte
\newlength{\summht}
\setlength{\summht}{0.18\wholeboxht}%5cm Zusammenfassung
\newlength{\cgridht}
\setlength{\cgridht}{\wholeboxht-\summht}
\newlength{\cgridwd}
\setlength{\cgridwd}{\wholeboxwd-\cuewd}
\newlength{\xorig}
\newlength{\yorig}
\setlength{\xorig}{0cm}
\setlength{\yorig}{0cm}

\begin{document} 
	% Seite 1
	\begin{center}
		\begin{tikzpicture} 
		% Notizen kariert: 0.5cm
		\draw[step=.5cm,gray!60,thin] (\cuewd,\summht) grid (\wholeboxwd,\wholeboxht);
		%% Optional: Notizen kariert: 1mm
		%\draw[step=.1cm,gray!30,thin] (\cuewd,\summht) grid (\wholeboxwd,\wholeboxht);      
		% Summary, top:
		\draw [line width=.2pt,gray!60,thin] (\xorig,\summht) -- (\wholeboxwd,\summht);  
		% Grid, left:
		\draw [line width=.2pt,gray!60,thin] (\cuewd,\summht) -- (\cuewd,\wholeboxht);
		% Draw the big box:
		\draw [line width=.2pt,gray!60,thin](\xorig,\yorig) rectangle (\wholeboxwd,\wholeboxht);
		\node[anchor=west] at (0.25,\wholeboxht-1em) {\textbf{Stichworte/Fragen}};
		\node[anchor=west] at (0.25,\summht-1em){\textbf{Zusammenfassung}}; 
		\node[anchor=west,fill=white] at (\cuewd+1em,\wholeboxht-1em) {\textbf{Notizen}};
		\node[anchor=west] at (0.25,\wholeboxht+2em){%
			\parbox[t]{\topwd}{\textbf{Dozent/Quelle:} \hspace{2cm} \textbf{Thema:} \hfill \textbf{Datum:}}%
		};
		\end{tikzpicture} 
	\end{center}
	
\end{document}